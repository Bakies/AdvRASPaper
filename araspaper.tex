\documentclass[12pt]{IEEEtran}
\usepackage{graphicx}
\usepackage{caption}
\usepackage{float}
\usepackage[style=ieee,backend=biber]{biblatex}
\usepackage[a4paper,margin=1in,footskip=0.25in]{geometry}
\usepackage[english]{babel}
\usepackage{pdfpages}

% uncomment for double spacing
%\usepackage{setspace}
%\doublespace

\usepackage[autostyle]{csquotes}
\addbibresource{araspaper.bib}
\usepackage{hyperref}
\hypersetup{
    colorlinks,
    citecolor=black,
    filecolor=black,
    linkcolor=black,
    urlcolor=black
}

\begin{document}

\title{Some Title}
\author{
Jon Bakies, Mitchell Dunn, Elias Kapetanopoulos, and Magdy Ellabidy \\
Department of Computer Science and Networking \\
Wentworth Institute of Technology \\
Boston, MA 02115, USA \\
bakiesj@wit.edu, dunnm8@wit.edu, kapetanopoulose@wit.edu, ellabidym@wit.edu
} 

\maketitle
\newpage
\clearpage

\section{Serial Over Lan}
\subsection{ser2net}
Ser2net is a daemon that opens a TCP or Telnet connection to the server's serial ports.
After connecting all of the devices' communication ports to the Raspberry Pi ser2net allows for a single telnet connection to each device.
There are some nuances that come along with this daemon.
Ser2net does not provide a service for SSH, so the communication between the user and the device will be unencrypted.
Although SSH would be preferred, a telnet connection is acceptable because in order to access the Serial over LAN (SoL) a user has to VPN into the network.
The only way to access the network is physically or via a secure connection.
Another nuance is the configuration.
The way ser2net opens a connection to the serial ports is by the device in the /dev directory.
The devices in the /dev directory are named by the order in which they are plugged in, meaning the needs to be an order in which they are plugged in.
\subsubsection{Installation}
ser2net is available on the apt package manger, so install is simple.

\textit{apt install ser2net}
\subsection{USB hub}
The USB hub is used to limit the wires run to the server room.
All of the devices in the pod are connected to the USB hub.
In order to constantly have the same device number, the devices need to be plugged in a specific configuration.



\section{Routers}

\subsection{Overview}
The routers and firewalls for this environment are configured with the OPNsense firewall software \cite{opnsense}. The software is highly configurable and made to provide state of the art software 

\subsection{Services}
\subsubsection{DHCP Server}
\subsubsection{OpenVPN Server}
\subsubsection{DNS Resolver}
\subsubsection{NAT}
\subsubsection{Firewall}

\subsection{Redundancy}
\subsubsection{CARP}
\subsubsection{pfSync}
\subsubsection{XMLRPC Sync}


\section{ESXi Hosts}
%\tableofcontents
%\newpage

%\newpage \section{References} \printbibliography[heading=none]
\end{document}
